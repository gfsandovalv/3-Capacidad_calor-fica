\section{Descripción experimental}
\subsection{Equivalente eléctrico}

Para la realización del experimento se usó un calorímetro dentro del cual se pone una masa de agua $m$, y dentro de esta se sumerge una resistencia de valor $R$ que está conectada en un circuito simple a una caida de pontecial $V$.
La figura \ref{fig:esquema_elec} muestra un esquema del montaje usado. Como instrumento de medición de temperatura, se usó un termistor conectado a un multímetro. 

Se realizaron dos tomas de datos, manteniendo las masas de agua cercanas y voltajes de $5$ y $10\,\si{\volt}$. En la tabla \ref{tab:takes_data} se muestran los datos de la masa de agua y el votaje aplicado a la resistencia 

\begin{table}[h]
    \centering
    \begin{tabular}{ccc}
        \toprule
        Toma & m (g) & V (V) \\
        \midrule
        1 & 117.0 & 10 \\
        2 & 109.8 & 5 \\
        \bottomrule
    \end{tabular}
    \caption{masa de agua y voltaje aplicado en cada toma de datos}
    \label{tab:takes_data}        
\end{table}

\subsection{Equivalente mecánico}

En el montaje usado (ver figura \ref{fig:montaje_mec}), la polea es un cilindro de aluminio el cual tiene incorporado un termistor que registra la temperatura del cilindro mismo.
El aparato utilizado tiene además un contador de vueltas. Los parámetros experimentales están resumidos en la tabla \ref{tab:mec_lab_datos}.

\begin{table}[h]
    \centering\begin{tabular}{ccc}
        \toprule
        & \text{Valor medido} & \text{Error} \\
        \midrule
        $M$ & $4324\,\si\gram$ & 1 \\
        $m$ & $310\,\si\gram$ & 1 \\
        $R$ & $2.68\,\si\cm$ & 0.001 \\
        $k_{\text{polea}}$ & $74.13\,\si\gram$ & 0.1 \\
        \bottomrule
    \end{tabular}
    \caption{Resumen de parámetros experimentales}
    \label{tab:mec_lab_datos}
\end{table}
